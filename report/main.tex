\documentclass[12pt, a4paper]{article}
\usepackage{geometry}
\parindent 0pt
\parskip 5pt
\pagestyle{plain}

\title{\textsc{Consensus}}
\author{}
\date{}

\begin{document}
\maketitle


\section{Motivation}

Achieving agreement among remote processes (where some may be faulty) is a
fundamental problem in distributed computing \cite{fischer1985impossibility}.
This problem is known as \textit{consensus} and lies ``at the core of many
algorithms for distributed data processing, distributed file management, and
fault tolerant distributed applications.'' \cite{fischer1985impossibility}


\section{Formal definition}
TBC


\section{Impossibility of consensus}

It has been shown that in an asynchronous system of processes, ``every protocol
for this problem has the possibility of nontermination, even with only one
faulty process.'' \cite{fischer1985impossibility}


\section{Relevant problems}

\subsection{Reliable Multicast}

Ensuring that all processes receive updates in the same order.

\subsection{Membership/Failure Detection}

Ensuring that every process has a local record containing every other process.
Failures should be detected and records should be updated.

\subsection{Leader Election}

Deciding on a leader among all processes, with all processes being aware of
who the leader is.

\subsection{Mutual Exclusion}

Ensuring that simultaneous access to a resource does not occur (exclusive
access).


\section{Paxos algorithm}

\subsection{Roles}

A process can take more than one role.

A process is \textit{persistent}---it cannot forget what it has accepted.

  \subsubsection{Proposer}

  \subsubsection{Acceptor}

  \subsubsection{Learner}

\subsection{Protocol}

\subsection{Issues}

  \subsubsection{Contention}

\subsection{Optimisations}



\section{Raft algorithm}

\subsection{Roles}

\subsection{Issues}

\subsection{Optimisations}


\section{Comparative analysis}


\section{Application}




\bibliographystyle{ieeetr}
\bibliography{main}

\end{document}
